%==============================================================================
% Sjabloon onderzoeksvoorstel bachproef
%==============================================================================
% Gebaseerd op document class `hogent-article'
% zie <https://github.com/HoGentTIN/latex-hogent-article>

% Voor een voorstel in het Engels: voeg de documentclass-optie [english] toe.
% Let op: kan enkel na toestemming van de bachelorproefcoördinator!
\documentclass{hogent-article}

\usepackage{xcolor}
\usepackage{soul}
\usepackage{graphicx}
\usepackage{float}
\usepackage{csquotes} % aangeraden bij biblatex

% (optioneel) als je afbeeldingen in een map zet, bv. images/
% \graphicspath{{./}{./images/}}

% -------------------- Kleurmarkeringen checklist --------------------
\newcommand{\markProbleem}[1]{{\sethlcolor{yellow!35}\hl{#1}}}
\newcommand{\markHoofd}[1]{{\sethlcolor{cyan!25}\hl{#1}}}
\newcommand{\markDVProb}[1]{{\sethlcolor{green!25}\hl{#1}}}
\newcommand{\markDVOpl}[1]{{\sethlcolor{magenta!18}\hl{#1}}}
% Invoegen bibliografiebestand
\addbibresource{voorstel.bib}

% Informatie over de opleiding, het vak en soort opdracht
\studyprogramme{Professionele bachelor toegepaste informatica}
\course{Bachelorproef}
\assignmenttype{Onderzoeksvoorstel}
% Voor een voorstel in het Engels, haal de volgende 3 regels uit commentaar
% \studyprogramme{Bachelor of applied information technology}
% \course{Bachelor thesis}
% \assignmenttype{Research proposal}

\academicyear{2025-2026}

\title{HouseHunt Alerts: een platform onafhankelijke immo-alerttool}

% TODO: Studentnaam en emailadres invullen
\author{Seppe Wauman}
\email{seppe.wauman@student.hogent.be}

% TODO: Medestudent
% Gaat het om een bachelorproef in samenwerking met een student in een andere
% opleiding? Geef dan de naam en emailadres hier
% \author{Yasmine Alaoui (naam opleiding)}
% \email{yasmine.alaoui@student.hogent.be}

% TODO: Geef de co-promotor op
\supervisor[Co-promotor]

% Binnen welke specialisatierichting uit 3TI situeert dit onderzoek zich?
% Kies uit deze lijst:
%
% - Mobile \& Enterprise development
% - AI \& Data Engineering
% - Functional \& Business Analysis
% - System \& Network Administrator
% - Mainframe Expert
% - Als het onderzoek niet past binnen een van deze domeinen specifieer je deze
%   zelf
%
\specialisation{Mobile \& Enterprise development}
\keywords{vastgoed, zoekalerts, web scraping, data-integratie}

\begin{document}

\begin{abstract}
In regio’s met een hoge vraag naar huurwoningen zoals Brussel, Gent, Leuven (Centrum/universiteitssteden) volgen kandidaat huurders nieuw online aanbod vaak intensief op via online huurplatformen. In de praktijk is het zoekproces echter gefragmenteerd over meerdere vastgoedplatformen. Dezelfde woning verschijnt geregeld op meerdere platformen die elk met eigen filters en platform gebonden alerts zitten. Wie meerdere platformen combineert krijgt daardoor snel te maken met dubbele meldingen en irrelevante meldingen als de filters te beperkt zijn. Deze bachelorproef onderzoekt welke ontwerpkeuzes in een centraal platform zoek en alert mechanisme het meest bijdragen aan het verminderen van dubbele en irrelevante alerts voor huurders in België. De focus ligt op het vergelijken van ontwerpkeuzes voormatching/filtering en deduplicatie van data. Een beperkte proof of concept wordt gebruikt als onderzoeksinstrument om verschillende varianten van matching en deduplicatiestrategieën toe te passen op huurdata van de twee grote Belgische huurplatformen (Immoweb, Immovlan). Via een praktijk gebaseerde evaluatie worden de varianten vergeleken op basis van meetcriteria zoals aandeel duplicaten, false positives, gemiste relevante panden en gebruikersfeedback over ervaren relevantie. Het resultaat bestaat uit onderbouwde conclusies en aanbevelingen over ontwerpkeuzes die meldingskwaliteit verbeteren binnen een gefragmenteerde en tijdsgevoelige huurmarkt.
\end{abstract}





\tableofcontents

% De hoofdtekst van het voorstel zit in een apart bestand, zodat het makkelijk
% kan opgenomen worden in de bijlagen van de bachelorproef zelf.
%==============================================================================
% Voorstel-inhoud — enkel de hoofdtekst (géén \documentclass, géén \begin{document})
%==============================================================================

% TODO (feedback): Is dit voorstel gebaseerd op een paper van Research Methods die je
% vorig jaar hebt ingediend? Heb je daarbij eventueel samengewerkt met een andere student?
% - Zo ja, haal dan de tekst hieronder uit commentaar en pas aan.
% - Zo nee, verwijder dit hele blok.

% \paragraph{Opmerking}
% Dit voorstel is gebaseerd op het onderzoeksvoorstel dat werd geschreven in het
% kader van het vak Research Methods dat ik (vorig/dit) academiejaar heb
% uitgewerkt (met medestudent VOORNAAM NAAM als mede-auteur).

%---------- Inleiding ---------------------------------------------------------

\section{Inleiding}
\label{sec:inleiding}

De Belgische huurmarkt is sterk verschoven naar online platformen: het grootste deel van het beschikbare aanbod wordt daar gepubliceerd. Voor kandidaat-huurders is het zoekproces bovendien tijdsgevoelig. Pandjes die aansluiten bij het budget en de locatievoorkeuren zijn vaak snel weg, waardoor regelmatig zoeken een voordeel oplevert. Tegelijk is het aanbod versnipperd. Dezelfde woning verschijnt niet zelden op meerdere platformen, elk met eigen filters, presentatie en meldingsmechanismen.

Veel platformen bieden zoekmeldingen of e-mailalerts, maar die zijn platformgebonden en laten vaak beperkte controle toe over criteria, transparantie en meldingskwaliteit. In de praktijk combineren kandidaat-huurders daarom meerdere websites: ze voeren dezelfde zoekopdracht opnieuw uit, houden verschillende alerts actief en vergelijken resultaten manueel. Dat kost tijd en zorgt voor meldingsruis. Dubbele meldingen ontstaan wanneer één woning op meerdere platformen wordt aangeboden; irrelevante meldingen ontstaan wanneer filters te grof zijn of wanneer platformen dezelfde voorkeuren niet op dezelfde manier ondersteunen.

Deze bachelorproef onderzoekt binnen die context welke ontwerpkeuzes in een centraal, platform-onafhankelijk zoek- en alertmechanisme het meest bijdragen aan het verminderen van dubbele en irrelevante meldingen. Het gaat daarbij niet om het afleveren van een product, maar om het vergelijken van ontwerpkeuzes voor normalisatie, matching/filtering en deduplicatie over meerdere bronnen heen.

Om die ontwerpkeuzes systematisch te kunnen analyseren, wordt een beperkte proof-of-concept gebruikt als onderzoeksinstrument. In deze demonstrator worden meerdere varianten van matching- en deduplicatiestrategieën toegepast op huurdata afkomstig van minstens twee Belgische huurplatformen. De evaluatie gebeurt aan de hand van meetbare criteria (onder meer aandeel duplicaten, false positives en gemiste relevante panden) en wordt aangevuld met feedback van testgebruikers over ervaren relevantie.

Om het onderzoek haalbaar en afgebakend te houden, wordt de scope beperkt tot:
\begin{itemize}
  \item huurpanden (geen koopmarkt);
  \item een beperkte selectie van grote Belgische huurplatformen (minstens twee bronnen binnen haalbare technische en juridische randvoorwaarden);
  \item een representatieve maar beperkte set zoekcriteria (bv. prijs, regio, type woning en aantal slaapkamers);
  \item een technische demonstrator die uitsluitend dient ter ondersteuning van het onderzoek en niet als productierijp systeem.
\end{itemize}


%---------- Probleemstelling, onderzoeksvraag en doelstellingen ----------------
\sloppy
\section{Probleemstelling, onderzoeksvraag en doelstellingen}
\label{sec:onderzoeksvraag}

\subsection*{Probleemstelling}

Voor kandidaat-huurders in België is het zoeken naar een geschikt huurpand sterk verspreid over meerdere online vastgoedplatformen. Wie snel wil reageren op nieuw aanbod, raadpleegt daardoor vaak meerdere websites en herhaalt dat proces regelmatig. In een markt waar geschikte panden snel verdwijnen, maakt die versnippering het zoeken tijdrovend en onoverzichtelijk.

Veel platforms bieden zoekmeldingen of e-mailalerts, maar die werken enkel binnen één platform en hebben vaak beperkte of weinig fijnmazige filtermogelijkheden. Wanneer kandidaat-huurders verschillende platformen combineren, ontstaan twee concrete problemen. Ten eerste komen dubbele meldingen vaak voor omdat dezelfde woning op meerdere platformen wordt gepubliceerd. Ten tweede leiden beperkte filters tot irrelevante meldingen: meldingen die niet (of slechts gedeeltelijk) aansluiten bij de zoekvoorkeuren van de gebruiker. Het gevolg is meldingsmoeheid en minder overzicht, waardoor belangrijke meldingen net gemist kunnen worden.

Een centraal, platform-onafhankelijk zoek- en alertmechanisme kan deze problemen in principe verminderen, maar het is vandaag onduidelijk welke ontwerpkeuzes daarvoor het meest doeltreffend zijn. In het bijzonder ontbreekt inzicht in de impact van verschillende aanpakken voor (i) normalisatie van data uit meerdere bronnen, (ii) matching en filtering, en (iii) deduplicatie over platformen heen. Dit onderzoek richt zich daarom op het vergelijken van die ontwerpkeuzes en het empirisch aantonen van hun effect op dubbele en irrelevante meldingen.

\subsection*{Onderzoeksvraag}

\begin{quote}
Welke ontwerpkeuzes in een centraal, platform-onafhankelijk zoek- en alertmechanisme voor de Belgische huurmarkt verminderen dubbele en irrelevante meldingen het meest, zonder dat relevante opportuniteiten onnodig verloren gaan?
\end{quote}

\subsection*{Deelvragen -- probleemdomein}

\begin{enumerate}
  \item Welke vastgoedplatformen zijn relevant voor kandidaat-huurders in België, en welke beperkingen vertonen de huidige platformgebonden zoek- en alertmechanismen op vlak van filtering, meldingskwaliteit en overlap tussen platformen?
  \item Welke minimale set van metadata is nodig om een platform-onafhankelijk zoek- en filterprofiel toe te passen (bv. prijs, locatie, type woning, slaapkamers, oppervlakte, beschikbaarheid), en in welke mate is die informatie consistent en publiek beschikbaar op de geselecteerde platformen?
  \item Welke technische en juridische randvoorwaarden bepalen de haalbaarheid van geautomatiseerde monitoring van de geselecteerde platformen (bv. \texttt{robots.txt}, gebruiksvoorwaarden, databankrechten en privacy), en welke beperkingen leggen die randvoorwaarden op aan dataverzameling en hergebruik?
\end{enumerate}

\subsection*{Deelvragen -- oplossingsdomein}

\begin{enumerate}
  \setcounter{enumi}{3}
  \item Welke architectuur is het meest geschikt om een centraal zoek- en alertmechanisme te realiseren (scheiding tussen dataverzameling, normalisatie, matching/deduplicatie en notificatie), en hoe beïnvloedt die architectuur de uitbreidbaarheid (nieuwe bron toevoegen) en onderhoudbaarheid (wijzigingen in bronstructuur)?
  \item Welke matching- en filterstrategieën leveren de beste balans tussen het beperken van irrelevante meldingen en het vermijden van gemiste opportuniteiten (bv. strikt filteren versus tolerant/scoring-gebaseerd), en welke parameters zijn daarbij bepalend?
  \item Welke deduplicatiestrategieën over meerdere bronnen heen verlagen het aandeel dubbele meldingen het sterkst met een aanvaardbaar risico op foutieve samenvoegingen (bv. exact match op adres versus combinatie van kenmerken zoals prijsband, locatie en woningkenmerken)?
  \item Welke evaluatiemethode en meetcriteria zijn geschikt om de impact van deze ontwerpkeuzes te vergelijken (bv. aandeel duplicaten, aantal false positives, aantal gemiste relevante panden, en gebruikersperceptie), en in welke mate stemmen de meetresultaten overeen met feedback van testgebruikers?
\end{enumerate}

\subsection*{Doelstellingen}

Dit onderzoek heeft als doel om:
\begin{itemize}
  \item de knelpunten in het huidige zoekproces van kandidaat-huurders in België te beschrijven, met focus op fragmentatie, tijdsdruk en meldingskwaliteit (dubbele en irrelevante meldingen);
  \item de relevante platformen en de minimaal noodzakelijke metadata te bepalen om zoekprofielen platform-onafhankelijk te kunnen toepassen;
  \item meerdere varianten van matching/filtering en deduplicatie te ontwerpen en systematisch te vergelijken binnen dezelfde context en dataset;
  \item de impact van die varianten empirisch te evalueren aan de hand van meetbare criteria (o.a. duplicaten, false positives en gemiste relevante panden), aangevuld met gebruikersfeedback;
  \item op basis van de resultaten onderbouwde conclusies en aanbevelingen te formuleren over doeltreffende ontwerpkeuzes en de beperkingen van een centraal, platform-onafhankelijk zoek- en alertmechanisme.
\end{itemize}


%---------- Literatuurstudie ---------------------------------------------------

\sloppy
\section{Literatuurstudie}
\label{sec:literatuurstudie}

Deze literatuurstudie ondersteunt de probleemstelling en onderzoeksvraag uit Sectie~\ref{sec:onderzoeksvraag}. 
De focus ligt daarom niet op het “bouwen van een tool”, maar op kennis die nodig is om ontwerpkeuzes voor een centraal, platform-onafhankelijk zoek- en alertmechanisme te vergelijken: (i) waarom fragmentatie en meldingsruis een probleem zijn, (ii) hoe je data uit meerdere bronnen vergelijkbaar maakt, (iii) welke matching- en deduplicatiestrategieën typisch gebruikt worden, en (iv) hoe je het effect daarvan meet.

\subsection*{Tijdsdruk en fragmentatie bij online woningzoeken}

Woningzoekgedrag is de voorbije jaren sterk online geworden. Studies tonen dat gebruikers vaak herhaaldelijk zoeken en filters gebruiken om snel een selectie te maken, en dat timing mee bepaalt welke aanbiedingen men effectief ziet \autocite{rae2015housingsearch}. 
Voor België wordt ook met online listingdata gewerkt om zoekactiviteit en marktgedrag te analyseren \autocite{vandenbergh2024belgiumsearch}. 
Deze literatuur ondersteunt het vertrekpunt van deze bachelorproef: in een gefragmenteerd landschap (meerdere platformen) stijgt de kans op overlap tussen resultaten en wordt het opvolgen van nieuw aanbod snel tijdrovend.

\subsection*{Meldingskwaliteit en meldingsruis}

Een alertmechanisme is pas nuttig als meldingen relevant blijven. Onderzoek rond notificaties toont dat frequente of storende meldingen de belasting verhogen en ertoe leiden dat gebruikers meldingen negeren of uitschakelen \autocite{ohly2023notifications}. 
Voor deze bachelorproef is dat vooral belangrijk om twee begrippen scherp te houden:
\begin{itemize}
  \item \textbf{dubbele meldingen}: dezelfde woning verschijnt meer dan één keer;
  \item \textbf{irrelevante meldingen}: meldingen die niet overeenkomen met het zoekprofiel (false positives).
\end{itemize}
Omdat strengere filters ook relevante panden kunnen missen, is het zinvol om in de evaluatie ook \textbf{gemiste relevante panden} (false negatives) mee te nemen.

\subsection*{Normalisatie en data-integratie over meerdere bronnen}

Wanneer data van meerdere platformen samenkomt, zijn velden en formaten zelden identiek. In data-integratie wordt dit doorgaans aangepakt door een beperkt intern schema te definiëren en brondata te normaliseren \autocite{rahm2001schemamatching}. 
Voor vastgoeddata betekent dit in de praktijk: een eenduidige representatie voor kernvelden zoals prijs, locatie, type woning en aantallen (bv.\ slaapkamers), en het expliciet documenteren van normalisatieregels. Die stap is nodig om matching en deduplicatie eerlijk te kunnen vergelijken.

\subsection*{Matching en deduplicatie (entity resolution)}

Het herkennen van dezelfde entiteit over meerdere bronnen heet in de literatuur \emph{entity resolution} of record linkage \autocite{getoor2012er, christen2012datamatching}. 
Een klassieke survey over duplicate detection bespreekt de typische bouwstenen: gelijkenisfuncties, blocking en de trade-off tussen nauwkeurigheid en efficiëntie \autocite{elmagarmid2007duplicate}. 
Voor vastgoedadvertenties is er bijna altijd ruis (variaties in tekst, prijsaanpassingen, onvolledige adressen). Daarom is het logisch om twee families van strategieën te onderscheiden die ook in dit onderzoek vergeleken worden:
\begin{itemize}
  \item \textbf{deterministisch}: exact match op sterke sleutels (bv.\ volledig adres) of strikte combinaties van kenmerken;
  \item \textbf{tolerant/scoring-gebaseerd}: meerdere kenmerken samen, met een drempel om records als duplicaat te beschouwen \autocite{fellegi1969recordlinkage}.
\end{itemize}
Die keuze heeft rechtstreeks impact op (a) het aantal duplicaten dat overblijft en (b) het risico op foutief samenvoegen.

\subsection*{Dataverzameling en randvoorwaarden}

Als er geen API beschikbaar is, kan crawling/scraping gebruikt worden, maar dat vraagt een voorzichtige aanpak: wijzigingen aan websites en “polite” crawling zijn realistische risico’s \autocite{olston2010webcrawling}. 
Daarnaast geeft het Robots Exclusion Protocol (robots.txt) richtlijnen die crawlers geacht worden te respecteren \autocite{rfc9309}. 
Ten slotte bepalen juridische kaders mee wat haalbaar is: databankbescherming (EU Databankenrichtlijn) en privacywetgeving (GDPR) zijn relevante randvoorwaarden bij hergebruik en opslag \autocite{eu1996databases, eu2016gdpr}. 
In deze bachelorproef worden die randvoorwaarden gebruikt om de selectie van bronnen en de scope van de proof-of-concept af te bakenen.

\subsection*{Evaluatie: meetcriteria voor “beter”}

Omdat de onderzoeksvraag draait rond impact op dubbele en irrelevante meldingen, moet de evaluatie dat ook zichtbaar maken. 
Voor het onderscheid tussen false positives en gemiste relevante items zijn precisie en recall standaardbegrippen uit information retrieval \autocite{manning2008ir}. 
Voor deduplicatie/ER worden gelijkaardige criteria gebruikt (bv.\ duplicaatpercentage na clustering en fouten door over- of onder-samenvoegen) \autocite{christen2012datamatching}. 
In dit onderzoek wordt daarom gemeten: (1) aandeel duplicaten, (2) false positives, (3) gemiste relevante panden, aangevuld met korte gebruikersfeedback over ervaren relevantie.

\subsection*{Synthese}

De literatuur ondersteunt dat woningzoeken tijdsgevoelig is en dat fragmentatie over platformen de kans op meldingsruis verhoogt \autocite{rae2015housingsearch}. 
Technisch wijst de literatuur erop dat normalisatie en entity resolution geen “one size fits all” hebben: keuzes rond matching en deduplicatie veroorzaken duidelijke trade-offs \autocite{getoor2012er, elmagarmid2007duplicate}. 
Daarom vergelijkt deze bachelorproef een beperkte set van varianten (strikt vs scoring-gebaseerd; simpel vs multi-feature deduplicatie) en evalueert ze die varianten met meetcriteria die zowel meldingsruis als gemiste opportuniteiten mee in beeld brengen.

%---------- Methodologie -------------------------------------------------------

\section{Voorgestelde methodologie}
\label{sec:methodologie}

Dit onderzoek volgt een toegepast onderzoeksdesign met een analyserend luik (probleemdomein) en een technisch luik (oplossingsdomein). Het plan wordt uitgewerkt als een reproduceerbaar stappenplan met concrete outputs per fase.

\subsection*{Fase 1 -- Probleemverkenning en requirements}

Doel: noden en verwachtingen van kandidaat-huurders expliciteren en vertalen naar requirements.
\begin{itemize}
  \item \textbf{Platformscan:} een beknopte vergelijking van alerts en filters op de gekozen huurplatformen (wat kan je instellen, welke velden worden getoond, welke beperkingen zijn er?).
  \item \textbf{Interviews:} semigestructureerde interviews met kandidaat-huurders (richtwaarde: 5--8). De interviewgids focust op zoekgedrag, frustraties (bv. dubbele listings, gemiste opportuniteiten), en gewenste criteria.
\end{itemize}
\textbf{Output:} requirementsdocument met functionele requirements, niet-functionele requirements en prioriteiten (MoSCoW).

\subsection*{Fase 2 -- Datainventaris en randvoorwaardenanalyse}

Doel: bepalen welke criteria en bronnen haalbaar zijn binnen technische en juridische randvoorwaarden.
\begin{itemize}
  \item \textbf{Datainventaris per platform:} welke attributen zijn publiek beschikbaar (prijs, locatie, type, slaapkamers, beschrijving, beschikbaarheid, \ldots), welke kwaliteit/consistentie hebben die velden, en hoe vaak verandert de structuur?
  \item \textbf{Randvoorwaardenanalyse:} per platform worden \texttt{robots.txt} en gebruiksvoorwaarden nagekeken, en worden databank- en privacyaspecten meegenomen op hoofdlijnen.
\end{itemize}
\textbf{Output:} tabel met per platform (a) beschikbare attributen, (b) haalbaarheid van extractie, (c) beperkingen/risico’s. Deze tabel bepaalt de uiteindelijke scope van de PoC (minstens twee bronnen binnen haalbare randvoorwaarden).

\subsection*{Fase 3 -- Ontwerp: datamodel, rule model en deduplicatie}

Doel: ontwerpkeuzes expliciteren zodat implementatie en evaluatie mogelijk worden.
\begin{itemize}
  \item \textbf{Intern datamodel:} definiëren van een uniform model voor huurpanden (kernattributen zoals prijs, regio, type, slaapkamers, bron-URL, publicatietijd).
  \item \textbf{Rule model:} starten met een beperkt maar uitbreidbaar model (bv. AND-criteria, prijsrange, regio, type, slaapkamers; later uitbreidbaar met extra criteria). Ontwerpdoel is dat gebruikers kunnen zien \emph{waarom} een listing matcht.
  \item \textbf{Deduplicatie:} ontwerpen van een matchingstrategie om dezelfde woning over bronnen heen te herkennen (bv. combinatie van locatie/adres-indicatoren, prijsband, woningkenmerken en broninformatie).
\end{itemize}
\textbf{Output:} datamodel (diagram), rule-representatie (bv. JSON-schema) en beschrijving van deduplicatielogica en matchingcriteria.

\subsection*{Fase 4 -- Implementatie proof-of-concept}

Doel: een werkende PoC bouwen die end-to-end het proces demonstreert.
De PoC omvat:
\begin{enumerate}
  \item een \textbf{collector} die periodiek listings ophaalt (binnen scope en randvoorwaarden);
  \item een \textbf{verwerkingspipeline} die normaliseert en dedupliceert;
  \item een \textbf{rule engine} die nieuwe listings vergelijkt met opgeslagen zoekprofielen;
  \item een \textbf{notificatiemechanisme} (bv. e-mail of webinterface) met instelbare frequentie (direct of gebundeld).
\end{enumerate}
\textbf{Output:} werkende PoC + technische documentatie + logging (doorlooptijd, aantal records, fouten) voor evaluatie.

\subsection*{Fase 5 -- Evaluatie}

Doel: haalbaarheid en meerwaarde beoordelen met meetbare criteria.
\begin{itemize}
  \item \textbf{Correctheid van matches:} testen met vaste zoekprofielen en een gecontroleerde set listings; rapportering met eenvoudige metrics (bv. \# correcte matches, \# false positives).
  \item \textbf{Deduplicatiekwaliteit:} meten van \% dubbele meldingen vóór/na deduplicatie en het aantal foutieve samenvoegingen (false merges) in een steekproef.
  \item \textbf{Bruikbaarheid:} kleinschalige gebruikersevaluatie met testgebruikers (taken: profiel aanmaken, criteria aanpassen, meldingen interpreteren) en korte feedback (bv. SUS of taak-succes + kwalitatieve opmerkingen).
\end{itemize}
\textbf{Output:} testresultaten (tabellen/grafieken), samenvatting van gebruikersfeedback, en conclusies + aanbevelingen.

%---------- Verwachte resultaten en bijdrage -----------------------------------

\sloppy
\section{Verwachte resultaten en bijdrage}
\label{sec:verwachte-resultaten}

De verwachte output van dit onderzoek is een onderbouwde proof-of-concept van HouseHunt Alerts: een platform-onafhankelijke tool die huurlistings uit meerdere bronnen verzamelt, normaliseert, dedupliceert en gebruikers verwittigt wanneer nieuwe panden voldoen aan hun criteria.

De verwachte bijdrage bestaat uit:
\begin{itemize}
  \item een overzicht van noden en frustraties van kandidaat-huurders in het digitale zoekproces;
  \item een datainventaris en haalbaarheidsoverzicht per platform (attributen, beperkingen, randvoorwaarden);
  \item een uniform datamodel en een eerste rule model voor configureerbare zoekprofielen;
  \item meetresultaten over (a) match-correctheid, (b) deduplicatie, (c) meldingsfrequentie en (d) bruikbaarheid;
  \item aanbevelingen voor verdere uitbouw (meer bronnen, richer rules, mobiele notificaties, ranking).
\end{itemize}

Hoewel het resultaat niet productierijp hoeft te zijn, moet de PoC aantonen dat het concept technisch uitvoerbaar is binnen de afgebakende scope, en dat het potentieel heeft om de tijdsbesteding aan manuele zoekacties te verminderen en de kwaliteit van meldingen te verhogen.

%---------- Verwacht resultaat en conclusie -----------------------------------

\sloppy
\section{Verwacht resultaat en conclusie}
\label{sec:verwacht_resultaat_conclusie}

Op basis van de probleemverkenning wordt verwacht dat kandidaat-huurders vooral hinder ondervinden van (a) fragmentatie over platformen, (b) tijdsdruk en (c) dubbele/irrelevante meldingen. Het verwachte resultaat is dat HouseHunt Alerts in een proof-of-concept aantoonbaar:
\begin{itemize}
  \item sneller nieuw aanbod detecteert dan manuele controle (binnen het gekozen monitoringinterval);
  \item minder dubbele meldingen verstuurt dankzij deduplicatie;
  \item voor testgebruikers duidelijk maakt waarom een pand matcht (transparantie van criteria);
  \item als bruikbaar wordt ervaren om het zoekproces te ondersteunen (op basis van taak-succes en korte feedback).
\end{itemize}

De conclusies van het onderzoek zullen afhangen van de evaluatieresultaten, maar beogen minstens:
(1) een uitspraak over technische haalbaarheid binnen de afgebakende scope en randvoorwaarden,
(2) inzicht in de kwaliteit van matching en deduplicatie,
en (3) aanbevelingen voor verdere uitbouw (meer bronnen, rijkere regels, andere notificatiekanalen of rankingmechanismen).


\clearpage
\appendix
\section{Planning}
\label{app:planning}

\begin{figure}[H]
  \centering
  \includegraphics[width=0.95\linewidth]{gantt.png}
  \caption{Planning bachelorproef (Gantt chart).}
\end{figure}


\printbibliography[heading=bibintoc,title={Referenties}]

\end{document}